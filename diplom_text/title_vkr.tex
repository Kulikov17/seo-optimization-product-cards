\begin{titlepage}
\newpage

{\setstretch{1.0}
\begin{center}
ПРАВИТЕЛЬСТВО РОССИЙСКОЙ ФЕДЕРАЦИИ\\
ФГАОУ ВО НАЦИОНАЛЬНЫЙ ИССЛЕДОВАТЕЛЬСКИЙ УНИВЕРСИТЕТ\\
«ВЫСШАЯ ШКОЛА ЭКОНОМИКИ»
\\
\bigskip
Факультет компьютерных наук\\
Образовательная программа «Прикладная математика и информатика»
\end{center}
}

\vspace{7em}

\begin{center}
{\bf ВЫПУСКНАЯ КВАЛИФИКАЦИОННАЯ РАБОТА}\\
%Выберите какой у вас проект
{\bf Исследовательский проект на тему:}\\
%{\bf Программный проект на тему:}\\
%{\bf Отчет о командном программном проекте на тему:}\\
{\bf Сжатие словарей для нейросетевого анализа исходных кодов программ}\\
\end{center}

\vspace{2em}

{\bf Выполнил студент: \vspace{2mm}}

{\setstretch{1.1}
\begin{tabular}{l@{\hskip 1.5cm}l}
группы \#БПМИ171, 4 курса & Гусев Андрей Алексеевич 
\end{tabular}}

% Обычно у вас есть один научный руководитель, и это человек, с которым вы работаете над проектом. Иногда по формальным причинам у вас будет руководитель (штатный сотрудник Вышки) и соруководитель (тот, с кем вы работаете), — об этом вам сообщит учебный офис (в случае с ВКР) или ЦППРиП (в случае с курсовым проектом). Также, если кто-то дополнительно вам помогал, то его можно указать как консультанта. 

%ваш официальный научник (из ВШЭ)
\vspace{1em}
{\bf Принял руководитель ВКР: \vspace{2mm}}

{\setstretch{1.1}
\begin{tabular}{l}
Чиркова Надежда Александровна\\
Научный сотрудник\\
Факультет компьютерных наук НИУ ВШЭ 
\end{tabular}}

% со-руководитель (если есть)
%\vspace{1em}
%{\bf Соруководитель: \vspace{2mm}}%это ваш официальный научник

%{\setstretch{1.1}
%\begin{tabular}{l}
%Петрова Надежда Александровна\\
%Инженер-исследователь\\
%ОАО Компания "Нейросети и деревья" 
%\end{tabular}}

% консультант (если есть)
%\vspace{1em}
%{\bf Консультант: \vspace{2mm}}%это ваш официальный научник

%{\setstretch{1.1}
%\begin{tabular}{l}
%Иванова Надежда Александровна\\
%Инженер-исследователь\\
%ОАО Компания "Нейросети и деревья" 
%\end{tabular}}

\vspace{\fill}

\begin{center}
Москва 2024
\end{center}

\end{titlepage}